\setuppagenumbering[location=none]

\definefontfamily [droid] [mono] [Droid Sans Mono]
\definefontfamily [mainface] [sans] [Arial]
\setupbodyfont [mainface, 12pt]

\setuplayout
    [
        topspace=20pt,
    ]

\definehead
    [title]
    [section]

\setuphead
    [title]
    [
        style={\ss\bfd},
        before={\blank[none]},
        after={\blank[none]},
        number=no,
        page=no,
        header=empty,
    ]

\setuphead
    [section]
    [
        number=no,
    ]

\starttext
\title{Hello world}

\blackrule[width=\textwidth, height=1pt]
\blank[2*big]

HELLo world!

\startsection
    [
        title={\bf Wejście},
        bookmark={Wyjście},
    ]

W pierwszej linii wejścia znajduje się jedna liczba całkowita $$n\hskip .5em(-10^3 \leq n \leq 10^3)$$.
\stopsection

\startsection
    [
        title={\bf Wyjście},
        bookmark={Wyjście},
    ]

Napisz $$n$$ linii zawierających \quotation{Hello world}.
\stopsection

\startsection
    [
        title={\bf Przykład},
        bookmark={Przykłady},
    ]
\definefontfamily [mainface] [serif] [Calibri]
\setupbodyfont [mainface, 12pt]
\startcolumns[n=2]
\setupbodyfont [droid, 12pt]
\framed[frame=on, align=normal, width=\textwidth]{
    2
}{\setupbodyfont[mainface, 9pt] Wejście dla testu $$hel0a$$}

\framed[frame=on, align=normal, width=\textwidth]{
    Hello world\\
    \hskip .515em Hello world
}{\setupbodyfont[mainface, 9pt] Wyjście dla testu $$hel0a$$}
\stopcolumns

\startcolumns[n=2]
\setupbodyfont [droid, 12pt]
\framed[frame=on, align=normal, width=\textwidth]{
    6
}{\setupbodyfont[mainface, 9pt] Wejście dla testu $$hel0b$$}

\framed[frame=on, align=normal, width=\textwidth]{
    Hello world\\
    \hskip .515em Hello world\\
    \hskip .515em Hello world\\
    \hskip .515em Hello world\\
    \hskip .515em Hello world\\
    \hskip .515em Hello world
}{\setupbodyfont[mainface, 9pt] Wyjście dla testu $$hel0b$$}
\stopcolumns

\stopsection
\stoptext